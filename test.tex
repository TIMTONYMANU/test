\documentclass[zihao=5, a4paper, UTF8]{ctexart} % 【要求1】全局基础字号设为五号字 (10.5pt)

% ==================== 数学与排版宏包 ====================
\usepackage{amsmath, amssymb, amsfonts}

% ==================== 【要求4】页面与边距 ====================
% A4纸宽21cm,上下2cm,左右2.5cm。正文有效宽度恰好为 16cm。
% 要求页脚距离页边距1.5cm,所以 footskip (正文底部到页脚的距离) = 2cm - 1.5cm = 0.5cm。
\usepackage{geometry}
\geometry{
    a4paper,
    top=2cm,
    bottom=2cm,
    left=2.5cm,
    right=2.5cm,
    footskip=0.5cm 
}

% ==================== 【要求3】行间距 ====================
\usepackage{setspace}
\setstretch{1.5} % 严格对应 Word 中的 1.5 倍行距

% ==================== 【要求1 & 2】字体与字间距 ====================
\usepackage{fontspec}

% 【要求2计算】:Word中的加宽0.7磅。五号字是10.5pt,0.7/10.5 ≈ 6.67%。
% 设置 LetterSpace=6.67 完美模拟 Word 的“加宽0.7磅”。
% 【要求1】:英文和数字为 Times New Roman
\setmainfont{Times New Roman}[LetterSpace=12.67] 

% 【要求1】:中文为纯正宋体 (开启 AutoFakeBold 让标题可以正常加粗)
\setCJKmainfont{SimSun}[LetterSpace=6.67, AutoFakeBold=true]

% 【要求1】:公式字号设为 12 磅 (语法: 基础字号, 公式字号, 上标字号, 双重上标)
\DeclareMathSizes{10.5}{12}{9}{7} 

% ==================== 【要求5】标点符号强制宋体 ====================
% 强制将英文半角的 ( ) , . : ; 划分为中文字符类,从而让它们不再使用Times,而是使用宋体渲染。
\xeCJKDeclareCharClass{CJK}{"0028, "0029, "002C, "002E, "003A, "003B}

% ==================== 【要求6】完美模拟 Word 制表位 ====================
% 因为正文宽度正好是 16cm。
% 左对齐(0cm)、居中(8cm处)、右对齐(16cm最右侧)。
\newcommand{\wordtab}[3]{%
    \noindent
    \makebox[0pt][l]{#1}% 左侧起点
    \hspace*{8cm}\makebox[0pt][c]{#2}% 绝对 8cm 处居中
    \hfill\makebox[0pt][r]{#3}% 绝对 16cm 处右对齐
    \par
}

% 消除默认的段落缩进(可选,为了贴合试卷排版)
\setlength{\parindent}{0pt}

\begin{document}

% —————— 这里为你演示【要求6】的制表位效果 ——————
%\wordtab{姓名:\underline{\hspace{2cm}}}{班级:\underline{\hspace{2cm}}}{考号:\underline{\hspace{2cm}}}
%\vspace{1em}
% ————————————————————————————————————————————————

\begin{center}
    \Large\textbf{2023 年量子物理思维挑战夏令营原创模拟试题}\\[1em]
    \LARGE\textbf{第一套}\\[1em]
    \normalsize (2023 年 7 月 16 日上午 8:30-11:30)
\end{center}

\begin{center}
    \textbf{考生必读}
\end{center}
1. 考生考试前请务必认真阅读本须知.\\
2. 本试题共 5 页, 总分为 320 分.\\
3. 如遇试题印刷不清楚的情况, 请务必向监考老师提出.\\
4. 需要阅卷老师评阅的内容一定要写在答题纸上; 写在试题纸上和草稿纸上的解答一律不能得分.

\vspace{2em}

\textbf{一、(40 分)}如图所示, 研究一个简化的不倒翁刚体模型, 不倒翁下半部分为质量为 $m$ 半径为 $R$ 的匀质薄半球壳, 上半部分为质量也为 $m$ 的匀质圆锥薄板, 已知该不倒翁系统的质心仍在半球的球心 $O$.

(1)求圆锥的半顶角 $\theta$ 以及不倒翁绕过球心某一水平直径的转动惯量 $I$.

(2)将该不倒翁竖直静止放置在一粗糙水平面上, 在 $A$ 点突然沿半径方向给不倒翁一个水平冲量 $J=2m\sqrt{gR}$, 若不倒翁在 $B$ 点着地时恰好到达纯滚动. 在 $B$ 点着地后, $B$ 点是否会再移动? 如果会, 求 $B$ 点开始移动时 $AOB$ 线与竖直面的夹角, 如果不会求 $BC$ 同时着地时不倒翁转动的角速度. 假设不倒翁和地面的动摩擦系数和静摩擦系数相同.

(3)从(2)可知, 显然该系统并非是真正的不倒翁, 为使其成为真正的不倒翁系统(即在粗糙地面上, 不倒翁在外力作用下可以使得半圆锥面 $BC$ 着地, 但在释放外力后系统会回到图示 $CO$ 为竖直方向的位置), 可以在不倒翁最低点 $D$ 处粘住一个密度很大的物体, 可以视为质点, 求该物体的最小质量 $M$, 以及在该质量下系统在足够粗糙水面内做小振动的周期.

\vspace{1em}

\textbf{二、(40 分)}如图所示, 竖直平面内存在质量为 $2m$ 弯成的半圆形匀质光滑铁丝, 初始时, 铁丝以角速度 $\omega_0$ 在空间内绕 $OC$ 轴转动, 在 $A$ 和 $B$ 点正上方的 $h$ 处静止放置两个质量为 $m$ 两个小球 $M$ 和 $N$, 小球可以无摩擦的穿过铁丝滑动. 小球 $M$ 和 $N$ 同时从静止开始释放, 已知两小球下落后刚好能分别从 $A$ 和 $B$ 穿过铁丝.

(1)小球从 $A$ 和 $B$ 穿过铁丝后, 若铁丝在外力作用下仍能以恒定的角速度 $\omega_0 = \sqrt{\frac{4g}{R}}$ 旋转, 求其中一个小球能到 $C$ 点的条件以及小球到达最低点时与铁丝的相互作用力大小.

(2)接(1), 为了保持铁丝能以该恒定的角速度旋转, 需要对铁丝施加外力, 求 $h=2R$ 时, 两小球在下降的过程中该外力功率大小的最大值.

(3)若铁丝只是能绕 $OC$ 轴无摩擦的转动, 除地面支持力外无其他外力左右. 若小球从任意高度落下时两小球都能到达最低点 $C$, 求出初始时铁丝转动的角速度的最大值 $\omega_m$; 并求当 $h=2R$, $\omega_0 = \omega_m$ 小球在下降过程中的最大速率.

\vspace{1em}

\textbf{三、(40 分)}在月球和太阳的引力作用下, 海水每天两次的周期性涨落现象称为潮汐, 为简化模型, 忽略地球自转, 同时本题只讨论地月模式下的潮汐现象. 设月球质量为 $m$, 地球表面重力加速度为 $g$, 半径为 $R$, 地月质心距为 $r_m$.

(1)假设地球模型为在质量均匀分布的刚体球表面覆盖海水, 海水覆盖高度相对海水地球质量以及海水质量相比地球质量可忽略不计. 如图所示, 地球表面覆盖的海水会成为椭球形, 以地球质心为原点建立坐标系 $Oxyz$, 其中 $z$ 轴沿地月连线方向, 试求海水相对地球刚性表面的高度差, $\theta$ 为与 $z$ 轴夹角. 已知 $R=6371\text{km}, g=9.8\text{m}\cdot\text{s}^{-2}, m=7.35\times 10^{22}\text{kg}, r_m=3.84\times 10^8\text{m}, G=6.672\times 10^{-11}\text{N}\cdot\text{m}^2\cdot\text{kg}^{-2}$, 我们认为地球自转轴与地月公转轴平行, 给出海潮最大涨落幅度表达式并由此求出其数值解.

(2)实际地球模型并非刚体模型, 会发生“固体潮”, 即固体地球也会存在伸缩现象. 还是考虑简化 $\Delta r = f(\theta)$ 的地球模型, 假设地球为均匀、不可压缩、弹性椭球模型, 地球的切变模量为 $\mu$, 不考虑海水覆盖. 求在地月系统下, 地球发生固体潮后最大的涨落幅度的表达式. 为简化计算, 本文中可设地球的平均密度 $\rho$ 已知, 且最终答案由 $\rho, G, m, R, \mu, r_m$ 表示.

提示 1: 质量为 $m$, 半径为 $R$、切变模量为 $\mu$ 的均匀不可压缩弹性球变成一个偏心率为 $e$ 的、长旋转椭球时, 表面方程、弹性势能和引力自能分别为:
\[ r = R + \frac{1}{6}e^2 R (3\cos^2\theta - 1) \]
\[ V_e = \frac{38}{225}\pi \mu e^4 R^3 \]
\[ U_g = -\frac{3}{10}\frac{Gm^2}{R}\frac{(1-e^2)^{1/3}}{e} \ln \frac{1+e}{1-e} \]

提示 2: 对于旋转椭球体 $V = \{(x,y,z)|\frac{x^2+y^2}{b^2} + \frac{z^2}{a^2} \le 1\}$ 定义偏心率 $e = \frac{\sqrt{a^2-b^2}}{a}$:
\[ \iiint_V x^2 \text{d}V = \frac{4\pi a b^4}{15} \quad , \quad \iiint_V z^2 \text{d}V = \frac{4\pi a^3 b^2}{15} \]

\vspace{1em}

\textbf{四、(40 分)}当任何点电荷(例如电子)有正或负的加速度时, 都会以电磁辐射的形式释放能量.

(1)已知, 对于非相对论电子, 其电磁辐射的总功率可以由拉莫尔公式求出, 在高斯单位制下, 其基本形式为: $P = k e^\alpha \dot{v}^\beta c^\lambda$, 其中 $k$ 为常系数, $e$ 为电子电荷量, $\dot{v}$ 为加速度, $c$ 为光速, 求 $\alpha, \beta, \lambda$.

(2)为了确定(1)中常系数 $k$ 的值, 考虑一简化模型: 一个电偶极矩 $p$ 的电偶极子, 其大小做圆频率为 $\omega$ 简谐振动, 方向固定, 该电偶极子将辐射电磁波. 可以证明, 采用高斯单位制时, 在足够远的区域观察时, 该电磁波近似为球面波, 相应的电矢量和磁矢量分别为:
\[ \vec{E} = \frac{1}{c^2 R^3} \vec{R} \times (\vec{R} \times \ddot{\vec{p}}) \]
\[ \vec{H} = -\frac{1}{c^2 R^2} (\vec{R} \times \ddot{\vec{p}}) \]
\[ \vec{S} = \frac{c}{4\pi} \vec{E} \times \vec{H} \]
式中 $\vec{R}$ 为观察点的位置矢量, 根据此模型求拉莫尔公式中常数 $k$.

(3)由此我们通过建立非常简单的模型——磁偶极模型来讨论解释脉冲星某些观测特性, 此模型特点是强调脉冲星的辐射如何由旋转的中子星的动能加以驱动. 其模型的物理图像是: 设中子星以一定的频率在真空中均匀旋转并具有磁偶极距, 同时假定这一旋转足够慢以致由旋转引起的非球形畸变可加以忽略. 已知在恒星的磁极处的纯偶极场为 $B_p$, 其大小与磁偶距 $m$ 大小的关系可表示为
\[ |m| = \frac{B_p R^3}{2} \]
磁矩作简谐运动时的辐射功率为
\[ \dot{E} = \frac{2}{3c^3}|\ddot{\vec{m}}|^2 \]
其中 $R$ 是恒星的半径, 如此的结构当无限远处看去存在一个随时间变化的偶极矩, 因此其向外辐射的辐射能量的速率可以类比(2)得到. 若某时刻观测到某蟹状星云脉冲星半径为 $R$, 转动惯量为 $I$, 恒星的转动的角速度为 $\Omega$, 磁极处的纯偶极场大小为 $B_p$, 取向与旋转轴呈 $\alpha$ 角. 求此脉冲星的特征年龄(即假设脉冲星以当前瞬时角加速度减速至静止的时间).

\vspace{1em}

\textbf{五、(40 分)}磁致冷技术基本原理在给磁性材料施加磁场时, 磁矩将按磁场方向齐排列(磁熵变小), 然后再撤去磁场, 使磁矩的方向变为紊乱(磁熵变大), 这时磁体从周围吸收热量, 通过热交换使周围环境的温度降低, 达到致冷的目的. 本题讨论将绝热去磁制冷的基本原理(adiabatic demagnetization refrigeration, ADR).

(1)假设某种用来磁致冷的材料为顺磁体, 其内部分子固有磁矩已知为 $m_0$, 将该顺磁体置于外磁场中, 由于其分子的磁矩排列将发生变化而产生磁化; 磁化程度用单位体积内的磁矩(磁化强度) $\vec{M}$ 描述, $\vec{M} = \chi \vec{H}$, 其中 $\chi$ 为磁化率, $\vec{H} = \frac{\vec{B}}{\mu_0} - \vec{M}$ 为磁场强度, $\vec{B}$ 为磁感应强度, $\mu_0$ 为真空磁导率. 在常温 $T$ (绝对温度)下, 设磁感应强度 $\vec{B}$ 是均匀的, 该材料顺磁体的分子数密度为 $n$, $\frac{m_0 B}{kT} \ll 1$, 不考虑磁化效应产生的磁场. 试利用玻尔兹曼统计证明顺磁体的磁化率满足 $\chi = \frac{C}{T}$, 并给出 $C$ 的表达式, $C$ 为该材料有关的常量.

(2)利用(1)中的材料构建的理想 ADR 循环如图所示, 循环过程如下:\\
① $ab$ 段, 等温磁化: 磁热材料被预加热致温度 $T_1$ 后($a$ 点), 此时外加磁场为 $H_a$, 保持热开关闭合, 施加磁场并控制磁化速率使其等温磁化, 直至达到最大磁场 $H_b$ ($b$ 点);\\
② $bc$ 段, 绝热去磁: 断开热开关, 逐渐减小磁场, 使工质在绝热条件下降温至目标温度 $T_2$, 此时仍剩余一定磁场($c$ 点);\\
③ $cd$ 段, 等温去磁: 在有负荷状态下, 继续去磁并控制去磁速率, 以温度 $T_2$ 进行等温制冷, 直至磁场为 $0$ ($d$ 点);\\
④ $da$ 段, 绝热磁化: 当去磁完全后, 需对材料磁化再生, 保持热开关断开, 施加磁场, 使工质在绝热条件下升温至等温磁化温度($a$ 点), 从而进入下一循环.

已知该磁性材料内能只与温度相关, 求上述循环过程中在 $C$ 点时的磁场 $H_c$, 并求出 ①③ 过程吸放热(以吸热为正), 并由此求出该过程的制冷系数.\\
提示: 在一个无限小的准静态过程中, 外界对磁介质做功可表示为:
\[ dW = \mu_0 H \text{d}M \]
式中, $H$ 为外磁场, $M$ 为介质的总磁矩.

\vspace{1em}

\textbf{六、(60 分)}某品牌追光灯内部结构如图所示: 由椭球反射镜, 球面反射镜和透镜组成. 光源既位于椭球面反射镜焦点 $F_1$ 上, 又与球面反射镜的球心重合, 而椭球面的另外一焦点又位于平凸透镜的焦点. 已知椭球面反射镜的离心率为 $e$, 平凸透镜的孔径足够大, 能使得光源发出的所有光线通过透镜, 球面反射镜半径足够大. 其他角度几何结构如图(a)所示, 角度为已知量. 灯泡的放光方向如图(b)所示, 图中只画出了子午面内上半部分放光方向的分布, 图中的 $-60^\circ \sim +60^\circ$ 内放光亮度一样, 在 $-90^\circ \sim -60^\circ$ 和 $+60^\circ \sim +90^\circ$ 范围内不发光.

(1)将可变光阑的孔径从大向小调节, 当其半径为 $r_1$ 时, 灯泡发射的光刚好全部通过光阑, 当其半径为 $r_2$ 时, 灯泡发射的光刚好全部不能通过光阑, 求 $r_1$ 和 $r_2$ 的比值.

(2)当可变光阑可以使得灯源的光全部通过时, 若已知平凸透镜的焦距为 $f$, 椭球面反射镜的离心率为 $e=0.5$, 且此时在接收光屏上最亮的地方亮度为 $I_m$, 求屏上亮度与到光轴距离 $r$ 之间的关系 $I(r)$.

\vspace{1em}

\textbf{七、(60 分)}在实验室参考系中, 质量(静质量)为 $M$ 的物体以一定的速度 $v$ 撞击质量(静质量)为 $m < M$ 的物体, 两物体均可视为质点, 二者之间的碰撞是完全弹性的, 忽略其他外力, 在 $M$ 和 $m$ 的质心系中, $m$ 出射方向是各向同性的(出射粒子在各个方向上出现的概率相等). 而在实验室系中, $m$ 的出射方向与 $M$ 入射的夹角为 $\theta$, 则:

(1)非相对论条件下, 试求碰后 $m$ 出射方向的角分布 $f(\theta) = \frac{\text{d}P}{\text{d}\theta}$ (按其出射角的概率分布), 并求出最概然方向(即出射粒子最可能出现的 $\theta$ 方向);

(2)相对论条件下, 试求碰后 $m$ 出射方向的角分布 $f(\theta) = \frac{\text{d}P}{\text{d}\theta}$ (按其在实验系下出射角的概率分布), 并求出最概然方向. 计算最概然方向时给出 $M=2m, v=0.6c$ 时的结果.

(3)在相对论条件下试求碰后 $m$ 总能量随能量的概率分布, 注意写出能量分布范围.

\end{document}
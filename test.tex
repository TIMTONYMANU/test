\documentclass[zihao=5, a4paper, UTF8]{ctexart} % 【要求1】全局基础字号设为五号字 (10.5pt)
\usepackage{xeCJK}
\usepackage{enumitem}
\usepackage{amsmath, amssymb, amsfonts}
\usepackage{geometry}
\usepackage{setspace}
\usepackage{fontspec}


% 【要求1】:英文和数字为 Times New Roman
\setmainfont{Times New Roman}

% 【要求1】:中文为纯正宋体 (开启 AutoFakeBold 让标题可以正常加粗)
\setCJKmainfont{SimSun}[AutoFakeBold=true]

\AtBeginDocument{
  \xeCJKsetup{
    CJKglue=\hskip 1.7pt
  }
}

% 【要求2】:公式字号设为 12 磅 (语法: 基础字号, 公式字号, 上标字号, 双重上标)
\DeclareMathSizes{10.5}{12}{9}{7} 

% ==================== 【要求3】行间距 ====================

\setstretch{1.5} % 严格对应 Word 中的 1.5 倍行距

% ==================== 【要求4】页面与边距 ====================
% A4纸宽21cm,上下2cm,左右2.5cm。正文有效宽度恰好为 16cm。
% 要求页脚距离页边距1.5cm,所以 footskip (正文底部到页脚的距离) = 2cm - 1.5cm = 0.5cm。

\geometry{
    a4paper,
    top=2cm,
    bottom=2cm,
    left=2.5cm,
    right=2.5cm,
    footskip=0.5cm 
}

% ==================== 【要求5】标点符号强制宋体 ====================
% 强制将英文半角的 ( ) , . : ; 划分为中文字符类,从而让它们不再使用Times,而是使用宋体渲染。
\xeCJKDeclareCharClass{CJK}{"0028,"0029,"002C, "002E, "003A, "003B}

% 消除默认的段落缩进(可选,为了贴合试卷排版)
\setlength{\parindent}{0pt}

%页码设置
\usepackage{fancyhdr}
\usepackage{lastpage}

\pagestyle{fancy}
\fancyhf{}

% 页码格式:第x页,共x页

\fancyfoot[C]{
{\fontsize{9pt}{11pt}\selectfont
第 \textnormal{\thepage} 页,共 
\textnormal{\pageref{LastPage}} 页}
}
\renewcommand{\headrulewidth}{0pt}
\renewcommand{\footrulewidth}{0pt}
% 页脚距离底部(非常重要,和你 geometry 兼容)
\setlength{\footskip}{0.5cm}

%放置标点单独一行
\XeTeXlinebreaklocale "zh"
\XeTeXlinebreakskip = 0pt plus 1pt

\begin{document}
\setlength{\baselineskip}{18.9pt}
\begingroup
\setstretch{1.0}  % 标题区域单倍行距

% ===== 主标题 =====
\vspace{3pt}
{\centering
\fontsize{16pt}{18pt}\selectfont\bfseries
2023年量子物理思维挑战夏令营原创模拟试题\par}
\vspace{13pt}

% ===== 副标题 =====
\vspace{3pt}
{\centering
\fontsize{16pt}{18pt}\selectfont\bfseries
第一套\par}
\vspace{13pt}

% ===== 日期 =====
\vspace{3pt}
{\centering
\fontsize{10pt}{14pt}\selectfont\bfseries
(2023 年 7 月 16 日上午 8:30-11:30)\par}
\vspace{10.5pt}

% ===== 考生必读 =====

{\centering
\fontsize{12pt}{16pt}\selectfont\bfseries
考生必读\par}
\vspace{10.5pt}

\endgroup
% 使用 wide 参数可以让编号后的文字换行时不再产生额外的悬挂缩进
% leftmargin 控制整体缩进(比如缩进 2 个字符)
% labelindent 控制编号距离左边界的距离

\begin{enumerate}[
    label={\arabic*、}, 
    wide=2em,        % 核心参数:让编号和正文在同一水平线下起步,换行不缩进
    labelindent=0pt,
    nosep
]
    \item \textbf{考生考试前请务必认真阅读本须知。}
    \item \textbf{本试题共 5 页,总分为 320 分。}
    \item \textbf{如遇试题印刷不清楚的情况,请务必向监考老师提出。}
    \item \textbf{需要阅卷老师评阅的内容一定要写在答题纸上;写在试题纸上和草稿纸上的解答一律不能得分。}
\end{enumerate}

\vspace{2em}
   
一、(\mbox40 分)如图所示, 研究一个简化的不倒翁刚体模型, 不倒翁下半部分为质量为 $m$ 半径为 $R$ 的匀质薄半球壳, 上半部分为质量也为 $m$ 的匀质圆锥薄板, 已知该不倒翁系统的质心仍在半球的球心 $O$.

\qquad(1)求圆锥的半顶角 $\theta$ 以及不倒翁绕过球心某一水平直径的转动惯量 $I$.

\qquad(2)将该不倒翁竖直静止放置在一粗糙水平面上, 在 $A$ 点突然沿半径方向给不倒翁一个水平冲量 $J=2m\sqrt{gR}$, 若不倒翁在 $B$ 点着地时恰好到达纯滚动. 在 $B$ 点着地后, $B$ 点是否会再移动? 如果会, 求 $B$ 点开始移动时 $AOB$ 线与竖直面的夹角, 如果不会求 $BC$ 同时着地时不倒翁转动的角速度. 假设不倒翁和地面的动摩擦系数和静摩擦系数相同.

\qquad(3)从(2)可知, 显然该系统并非是真正的不倒翁, 为使其成为真正的不倒翁系统(即在粗糙地面上, 不倒翁在外力作用下可以使得半圆锥面 $BC$ 着地, 但在释放外力后系统会回到图示 $CO$ 为竖直方向的位置), 可以在不倒翁最低点 $D$ 处粘住一个密度很大的物体, 可以视为质点, 求该物体的最小质量 $M$, 以及在该质量下系统在足够粗糙水面内做小振动的周期.

\vspace{1em}

二、(40 分)如图所示, 竖直平面内存在质量为 $2m$ 弯成的半圆形匀质光滑铁丝, 初始时, 铁丝以角速度 $\omega_0$ 在空间内绕 $OC$ 轴转动, 在 $A$ 和 $B$ 点正上方的 $h$ 处静止放置两个质量为 $m$ 两个小球 $M$ 和 $N$, 小球可以无摩擦的穿过铁丝滑动. 小球 $M$ 和 $N$ 同时从静止开始释放, 已知两小球下落后刚好能分别从 $A$ 和 $B$ 穿过铁丝.

\qquad(1)小球从 $A$ 和 $B$ 穿过铁丝后, 若铁丝在外力作用下仍能以恒定的角速度 $\omega_0 = \sqrt{\frac{4g}{R}}$ 旋转, 求其中一个小球能到 $C$ 点的条件以及小球到达最低点时与铁丝的相互作用力大小。

\qquad(2)接(1), 为了保持铁丝能以该恒定的角速度旋转, 需要对铁丝施加外力, 求 $h=2R$ 时, 两小球在下降的过程中该外力功率大小的最大值.

\qquad(3)若铁丝只是能绕 $OC$ 轴无摩擦的转动, 除地面支持力外无其他外力左右. 若小球从任意高度落下时两小球都能到达最低点 $C$, 求出初始时铁丝转动的角速度的最大值 $\omega_m$; 并求当 $h=2R$, $\omega_0 = \omega_m$ 小球在下降过程中的最大速率.

\vspace{1em}

三、(40 分)在月球和太阳的引力作用下, 海水每天两次的周期性涨落现象称为潮汐, 为简化模型, 忽略地球自转, 同时本题只讨论地月模式下的潮汐现象. 设月球质量为 $m$, 地球表面重力加速度为 $g$, 半径为 $R$, 地月质心距为 $r_m$.

\qquad(1)假设地球模型为在质量均匀分布的刚体球表面覆盖海水, 海水覆盖高度相对海水地球质量以及海水质量相比地球质量可忽略不计. 如图所示, 地球表面覆盖的海水会成为椭球形, 以地球质心为原点建立坐标系 $Oxyz$, 其中 $z$ 轴沿地月连线方向, 试求海水相对地球刚性表面的高度差, $\theta$ 为与 $z$ 轴夹角. 已知 $R=6371\text{km}, g=9.8\text{m}\cdot\text{s}^{-2}, m=7.35\times 10^{22}\text{kg}, r_m=3.84\times 10^8\text{m}, G=6.672\times 10^{-11}\text{N}\cdot\text{m}^2\cdot\text{kg}^{-2}$, 我们认为地球自转轴与地月公转轴平行, 给出海潮最大涨落幅度表达式并由此求出其数值解.

\qquad(2)实际地球模型并非刚体模型, 会发生“固体潮”, 即固体地球也会存在伸缩现象. 还是考虑简化 $\Delta r = f(\theta)$ 的地球模型, 假设地球为均匀、不可压缩、弹性椭球模型, 地球的切变模量为 $\mu$, 不考虑海水覆盖. 求在地月系统下, 地球发生固体潮后最大的涨落幅度的表达式. 为简化计算, 本文中可设地球的平均密度 $\rho$ 已知, 且最终答案由 $\rho, G, m, R, \mu, r_m$ 表示.

\qquad(3)提示 1: 质量为 $m$, 半径为 $R$、切变模量为 $\mu$ 的均匀不可压缩弹性球变成一个偏心率为 $e$ 的、长旋转椭球时, 表面方程、弹性势能和引力自能分别为:
\[ r = R + \frac{1}{6}e^2 R (3\cos^2\theta - 1) \]
\[ V_e = \frac{38}{225}\pi \mu e^4 R^3 \]
\[ U_g = -\frac{3}{10}\frac{Gm^2}{R}\frac{(1-e^2)^{1/3}}{e} \ln \frac{1+e}{1-e} \]

提示 2: 对于旋转椭球体 $V = \{(x,y,z)|\frac{x^2+y^2}{b^2} + \frac{z^2}{a^2} \le 1\}$ 定义偏心率 $e = \frac{\sqrt{a^2-b^2}}{a}$:
\[ \iiint_V x^2 \text{d}V = \frac{4\pi a b^4}{15} \quad , \quad \iiint_V z^2 \text{d}V = \frac{4\pi a^3 b^2}{15} \]

\vspace{1em}



\label{LastPage}
\end{document}